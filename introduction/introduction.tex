\chapter{Introduction}

\section{Motivation}

Prior to sending out their kicker, an NFL coach must weigh up the situation to determine whether the reward of conversion outweighs the risk of a miss. If the kick is from the 30 yard line of the windless Mercedes-Benz Superdome, history would suggest the kicker shouldn't have a problem. But set that line of scrimmage back another 30 yards and sprinkle in some late January snow and that decision might not be so straight forward. Would it be wise to instead punt the ball away because a missed field goal hands over precious field position? What about if playing it safe with a punt is out of the question, do we try move the chains with a fourth down conversion, perhaps making for a shorter kick later in the drive? A risk in itself.\\
On the other side of the field, staring down the barrel of dropping another 3 points, the defensive coach also has a decision to make. A psychological tactic known as "icing the kicker" has long been a go-to move for defensive coaches in a bid to unnerve the opposition kicker immediately prior to an attempt. This is done by using a timeout, however, of which only three are available per team per half. Which begs the question, is it worth it?\\
Either decision maker would benefit from a qualitative understanding of how these and various other environmental and psychological factors effect the likelihood of a successful attempt from our already beleaguered kicker.\\
That being said, not all analyses are made equal, and the trustworthiness of claims of causation rest on the make up of the data. In this study we compare two methods, one which employs "matching", which effectively reduces bias but increases variance by way of pruning observations, and another which doesn't, potentially increasing bias but reducing variance.

\section{Aims and Objectives}
In this study we aim to:
\begin{itemize}
    \item Quantify the influence of several factors to the likelihood of field goal conversion, employing the causal analysis technique of matching to do so.
    \item Repeat the process without matching.
    \item Compare the results of the two approaches, citing the variation in outcomes.
\end{itemize}

% Explaining the problem being solved.

\section{Description of the work}
Remove?
% Explaining what your project is meant to achieve, how it is meant to function, perhaps even a functional specification.


