\addcontentsline{toc}{chapter}{Abstract}

\begin{abstract}
From 2000 to 2019, 21.3\% of points scored in the NFL were from field goals. Furthermore, approximately 5\% of all attempts resulted in an immediate win or loss. 92\% of field goals are attempted on fourth down, whereby a miss results in the offense giving up precious field position. With all this in mind, one would argue that the decision to attempt a field goal should not be taken lightly. It's natural to hypothesise that there may be several factors that influence the likelihood of field goal conversion. Distance should be the first to spring to mind, however there are several environmental and psychological factors that may also play their part. Such environmental factors include temperature, wind, rain, snow, or even altitude. The psychological impact of the postseason, game situation, jeers from an away crowd, being iced\footnote{Icing refers to a timeout being called directly prior to a field goal attempt}, etc., may also have an effect. In this study, we aim to uncover the true causal effect of several of these factors by approximating a fully-blocked experiment using modern matching methods.
\end{abstract}